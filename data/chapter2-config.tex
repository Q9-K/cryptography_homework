% !Mode:: "TeX:UTF-8"
\chapter{RSA前置数学知识}

\section{互质}

如果两个正整数,除了 1 以外没有其他的公因数,则他们互质。比如,14 和 15 互质。注意,两个数构成互质关系,他们不一定需要是质数,比如 7 和 9。

关于互质有以下结论:
\begin{enumerate}
    \item 任意两个质数构成互质关系,比如13和61。
    \item 一个数是质数,另一个数只要不是前者的倍数,两者就构成互质关系,比如3和10。
    \item 如果两个数之中,较大的那个数是质数,则两者构成互质关系,比如97和57。
    \item 1和任意一个自然数是都是互质关系,比如1和99。
    \item p是大于1的整数,则p和p-1构成互质关系,比如57和56。
    \item p是大于1的奇数,则p和p-2构成互质关系,比如17和15。
\end{enumerate}

\section{欧拉函数}

考虑下面问题:

任意给定正整数n,请问在小于等于n的正整数之中,有多少个与n构成互质关系?(比如,在1到8之中,有多少个数与8构成互质关系?)

用$\varphi (n)$来表示上面问题的答案,有下面结论:

\begin{enumerate}
    \item 如果n=1,则 $\varphi(n) = 1$ 。因为1与任何数(包括自身)都构成互质关系。
    \item 如果n是质数,则 $\varphi(n) = n-1 $。因为质数与小于它的所有数都构成互质关系。
    \item 如果n是某质数p的某一个次方,即$n=p^k$,则 $\varphi(n) = p^k - p^{k-1}$。
    \item 如果n可以分解为两个互质的整数之积,即 $ n=a\times b $,则$\varphi(n) = \varphi(a) \times \varphi(b)$。
    \item 因为任意一个大于1的正整数,都可以写成有限个质数的积,即$ n = p_1^{k_1} \times p_2^{k_2} \times \cdots p_r^{k_r} $ 。所以,对于任意一个正整数n,都有$\varphi(n) = n \times (1-\frac{1}{p_1}) \times (1-\frac{1}{p_2}) \times \cdots \times (1-\frac{1}{p_k})$,其中$p_1,p_2,\cdots,p_k$是n的质因数。由结论3和结论4,经过推导可得$$ \varphi(n) = n(1-\frac{1}{p_1})(1-\frac{1}{p_2}\cdots(1-\frac{1}{p_r}))$$

          这就是欧拉函数的通用计算公式。
\end{enumerate}

\section{欧拉定理}

欧拉定理是一个非常重要的定理,它是RSA算法的理论基础。欧拉定理的内容如下:

如果a和n是互质的,那么$a^{\varphi(n)} \equiv 1 \pmod{n}$。

欧拉定理存在一个特殊情况:如果 p 是质数,而 a 不是 p 的倍数,此时 a 和 p 必然互质。因为$ \varphi(p) = p-1 $,所以
$$ a^{\varphi(p)} = a^{p-1} \equiv 1 \pmod{p}$$。
上面等式也称作\textbf{费马小定理}。

\section{模反元素}

如果两个正整数 a 和 n 互质,那么一定可以找到一个正整数 b,使得 ab - 1 被 n 整除。
$$ab \equiv 1 \pmod{n}$$
这个时候,b 就叫做 a 的 模反元素。可以用欧拉定理来证明,模反元素一定存在。
$$a^{\varphi(n)}= a\times a^{\varphi(n)-1} \equiv 1 \pmod{n}$$
所以,$a^{\varphi(n)-1}$就是a的模反元素。