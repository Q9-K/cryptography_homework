\chapter{RSA算法可靠性分析}

要保证RSA算法是可靠得,即是在已知公钥(n, e)得情况下能否保证私钥(n, d)得安全性。

因为 n 是公开的,所以私钥的安全本质上就是 d 的安全,那么有没有可能在得知 n 和 e 的情况下,推导得出 d?

\begin{enumerate}
    \item 因为$ ed \equiv 1 (mod \varphi(n))$,想知道d需要知道e和$\varphi(n)$
    \item 因为e是公开的,所以还需要知道$\varphi(n)$
    \item $\varphi(n)=(p-1) \times (q-1)$,计算$\varphi(n)$需要对n进行质数分解
\end{enumerate}

由上面的分析,d的安全性依赖于对n进行质数分解的难度。而现在\textbf{大整数的质数分解}是一个NP问题,目前没有多项式时间的算法可以解决,只能进行暴力破解。到2017年为止,世界上还没有任何可靠的攻击RSA算法的方式。只要密钥长度足够长,用RSA加密的信息实际上是不能被解破的。