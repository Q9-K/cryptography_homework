% !Mode:: "TeX:UTF-8"
\chapter{RSA算法步骤}

为了更具体的简介,我在下面步骤中以具体的数字进行说明。

\section{密钥生成}
\begin{enumerate}
    \item 随机选择两个不相等的质数p和q。在日常应用中,出于安全考虑,一般要求 n 换算成二进制要大于 2048 位。这里简单起见,我们选择两个较小的质数,$p=7,q=11$。
    \item 计算p和q的乘积$n = p \times q = 7$。
    \item 计算n的欧拉函数$\varphi(n) = (p-1) \times (q-1) = 60$。
    \item 选择一个数e使得e与$\varphi(n)$互质,且$1 < e < \varphi(n)$。在日常应用中,我们一般选择 65537。选择一个已知的数字不会降低 RSA 的安全性。这里我们选择$e=13$。
    \item 最后计算e相对于$\varphi(n)$的模反元素d。因为e和$\varphi(n)$互质,则
    $$ d = e^{\varphi[\varphi(n)]-1} $$
    由
    $$ e \times d \equiv 1 \pmod{\varphi(n)}$$
    得
    $$ e \times d = k\times \varphi(n) +1 $$
    于是
    $$ e \times d - k\times \varphi(n) = 1$$
    通过\textbf{扩展欧几里得算法}可以求解上面方程。此处解得$d = -23$,一般为了保证d为正数,我们可以加上$\varphi(n)$,即$d = 37$。
    \item 公钥是(n, e),私钥是(n, d)。这里算得,公钥是(77, 13),私钥是(77, 37)。
\end{enumerate}

\section{加密和解密}

\subsection{加密}
加密得过程是计算下面公式。

$$ m^e \equiv c \pmod{n}$$

上面算得公钥是(77, 13),则$ c = 7^{13} \pmod{77} = 35$。

故加密后的消息是35。

\subsection{解密}

由
$$ m^e \equiv c \pmod{n}$$
得
$$ c = m^e - k \times n$$
则
$$ c^d = (m^e - k \times n)^d = (m^e)^d + C_n^{1} \times (-k \times n) + \cdots + (-k \times n) ^d$$
由\textbf{二项式定理},除第一项外,每一项都是$k \times n$的倍数。

由
$$ e \times d \equiv 1 \pmod{\varphi(n)}$$
得
$$ e \times d = 1 + h \times \varphi(n) $$
于是
$$ c^d \bmod m = (m^e - k \times n)^d = (m^e)^d + C_n^{1} \times (-k \times n) + \cdots + (-k \times n) ^d \bmod m = m^ed \bmod m = m^{1 + h \times \varphi(n)} \bmod m$$

\begin{enumerate}
    \item m与n互质
    
    由欧拉定理,可得$$m^{\varphi(n)} \equiv 1 \pmod{n} $$
    则
    $$ m^{\varphi(n)} = kn + 1$$
    $$ (m^{\varphi(n)})^h = (kn+1)^h $$
    由二项式定理,可得
    $$ (m^{\varphi(n)})^h = t \times n + 1 $$
    即是
    $$ (m^{\varphi(n)})^h \equiv 1 \pmod{n} $$
    即是
    $$ m^{1 + h \times \varphi(n)} \equiv m \pmod{n}$$
    \item m与n不互质
     
    因为 n 是质数 p 和 q 的乘积,此时 m 必然为 kp 或者 kq。

    以 m = kp 为例,此时 k 必然与 q 互质。因为 n = pq,而 m < n,所以 k 必然小于 q,而 q 是一个质数,在小于 q 的数字当中所有数都与 q 互质。

    同时 kp 必然也与 q 互质,如果 kp 和 q 不互质,那么 kp 必然是 q 的倍数,因为 q 不存在其他因子,那么 kp 就是 n 的倍数,因为 n = pq,但是我们的前提是 m < n。

    因为 kp 和 q 互质,根据欧拉定理
    $$ (kp)^{q-1} \equiv 1 \pmod{q} $$
    所以
    $$ (kp)^{q-1} = t \times q + 1$$
    于是
    $$ [(kp)^{q-1}]^{h(p-1)} = (t \times q + 1)^{h(p-1)}$$ 
    同理根据二项式定理,右边展开除了 1 每一项都含有 q,所以可以得到
    $$ [(kp)^{q-1}]^{h(p-1)} \equiv 1 \pmod{q} $$
    于是
    $$ [(kp)^{q-1}]^{h(p-1)} \times kp =\equiv kp \pmod{q}$$
    即是
    $$ (kp)^{ed} \equiv kp \pmod{q} $$
    即是
    $$ (kp)^{ed} = kp + t \times q$$
    左边是 p 的倍数,右边 kp 是 p 的倍数,所以 tq 必然是 p 的倍数。而 q 是 p 互质的,因此 t 必然是 p 的倍数,我们记为 $t = t^{\prime}p$,代入得到
    $$ (kp)^{ed} = kp + t^{\prime}pq$$
    于是
    $$ m^{ed} \equiv m \pmod{n}$$
\end{enumerate}

于是可以用
$$ c^d \equiv m \pmod{n}$$
来进行解密。在上文中私钥是(77, 37),加密后的消息是35。
$$ 35^{37} = 7 \pmod{77}$$
所以解密后的消息是7。