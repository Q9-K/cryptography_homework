% !Mode:: "TeX:UTF-8"
\chapter{密码学发展史简介}

密码学早在公元前400多年就已经产生,人类使用密码的历史几乎与使用文字的时间一样长,密码学的发展大致可以分为 3 个阶段:
\begin{itemize}
    \item 1949年以前的古典密码学阶段
    \item 1949 年至 1975 年的近代密码学阶段
    \item 1976 年以后的现代密码学阶段
\end{itemize}

\section{古典密码学}
古典密码学的历史可以追溯到公元 400 年前,这一时期的密码学更像是一门艺术,其核心手段是代换和置换。代换是指明文中的每一个字符被替换成密文中的另一个字符,接收者对密文做反向替换便可恢复出明文;置换是密文和明文字母保持相同,但顺序被打乱。代换密码的著名例子有古罗马的凯撒密码(公元前1世纪)和法国的维吉尼亚密码(16世纪)。凯撒密码是对字母表中每个字母用它之后的第k个字母来代换,如,将“comeatnine”加密为“htrjfysnsj”(k=5)。但这种加密方式无法掩盖各字母的频率特征,易被破解。维吉尼亚密码相比之下提升了安全性,它的密钥通常是一个单词,如,“hear”,对于上述明文“comeatnine”,加密时将第1个字母后移8位(密钥“hear”的第一个字母h处于字母表第8位),第2个字母后移5位(密钥的第二个字母e处于字母表第5位),……,因此加密后的结果是“jsmvhxnzui”。

\section{近代密码学}
密码形成一门新的学科是在20世纪70年代,这是受计算机科学蓬勃发展刺激和推动的结果。快速电子计算机和现代数学方法一方面为加密技术提供了新的概念和工具,另一方面也给破译者提供了有力武器。计算机和电子学时代的到来给密码设计者带来了前所未有的自由,他们可以轻易地摆脱原先用铅笔和纸进行手工设计时易犯的错误,也不用再面对用电子机械方式实现的密码机的高额费用。

Arthur Scherbius于1919年设计出了历史上最著名的密码机—德国的Enigma机,,在二次世界大战期间, Enigma曾作为德国陆、海、空三军最高级密码机。Enigma机使用了3个正规轮和1个反射轮。这使得英军从1942年2月到12月都没能解读出德国潜艇发出的信号。转轮密码机的使用大大提高了密码加密速度,但由于密钥量有限,到二战中后期时,引出了一场关于加密与破译的对抗。首先是波兰人利用德军电报中前几个字母的重复出现,破解了早期的Enigma密码机,而后又将破译的方法告诉了法国人和英国人。英国人在计算机理论之父——图灵的带领下,通过寻找德国人在密钥选择上的失误,并成功夺取德军的部分密码本,获得密钥,以及进行选择明文攻击等等手段,破解出相当多非常重要的德军情报。

这一阶段真正开始源于香农在20世纪40年代末发表的一系列论文,特别是1949年的《保密系统通信理论》,把已有数千年历史的密码学推向了基于信息论的科学轨道。近代密码发展中一个重要突破是“数据加密标准”(DES)的出现。DES密码的意义在于,首先,其出现使密码学得以从政府走向民间,其设计主要由IBM公司完成,国家安全局等政府部门只是参与其中,最终经美国国家标准局公开征集遴选后,确定为联邦信息处理标准。其次,DES密码设计中的很多思想(Feistel结构、S盒等),被后来大多数分组密码所采用。再次,DES出现之后,不仅在美国联邦部门中使用,而且风行世界,并在金融等商业领域广泛使用。

\section{现代密码学}

1976 年,美国密码学家 Diffie 和 Hellman 提出了公钥密码的思想,提出“公钥密码”概念。此类密码中加密和解密使用不同的密钥,其中,用于加密的叫做公钥,用于解密的为私钥,这标致着现代密码学的诞生。

公钥密码体制的安全性均依赖于数学难题(大整数分解难题和离散对数求解难题)的困难性,但随着计算能力的不断增强和因子分解算法的不断改进,特别是量子计算机的发展,公钥密码安全性也渐渐受到威胁。目前,研究者们开始关注量子密码、格密码等抗量子算法的密码,后量子密码等前沿密码技术逐步成为研究热点。

目前实际应用的复杂算法仍以公钥算法为主,其中使用最广泛的就是 RSA 算法。1977 年罗纳德 · 李维斯特(Ron Rivest)、阿迪 · 萨莫尔(Adi Shamir)和伦纳德 · 阿德曼(Leonard Adleman)一起提出了 RSA 算法。

RSA 算法的可靠性由极大整数因数分解的难度决定。换言之,对一极大整数做因数分解愈困难,RSA 算法愈可靠。假如有人找到一种快速因数分解的算法的话,那么用 RSA 加密的信息的可靠性就肯定会极度下降。但找到这样的算法的可能性是非常小的。到 2017 年为止,还没有任何可靠的攻击 RSA 算法的方式。